%!TEX root = /Users/jakubkonka/Thesis/Thesis.tex
\chapter{Conclusions and Further Work}
\label{cha:conclusions}

\minitoc
\vspace{10mm}

\section{Conclusions} % (fold)
\label{sec:conclusions_conclusions}
The world of mobile communications is becoming increasingly diverse in terms of different wireless access technologies available: WiFi, 3G, and the cutting-edge 4G. In an environment of such diversity and heterogeneity, where each wireless access technology has its own distinct characteristics, intelligent network selection provides a resource efficient way of handling communications services by matching the services' required quality with the characteristics of a particular access technology.

To make full use of this increasingly diverse environment and increase the competition between network operators even further, the one-to-one mapping between network operators and subscribers need no longer hold. This allows the subscribers to seamlessly switch not only between different wireless access technologies belonging to one particular network operator, but also between network operators themselves. In this way, the subscriber, when requesting a bearer service, is given the option to select a network operator and a wireless access technology that best matches the required quality requirements of the service. It is not only to the benefit of the subscribers, however, since the integration of wireless access technologies will allow network operators for more efficient usage of network resources.

This thesis explored the economic aspects of intelligent network selection. The problem was studied within the context of Digital Marketplace---a market-based framework where network operators compete in a procurement auction-based setting for the right to transport the subscriber's requested service over their infrastructure. Since the Digital Marketplace was created with free market (or ``perfect'' competition) in mind, it is particularly well-suited towards the management of future wireless environment where wireless access is traded on a per connection basis. It is for this reason that this research explored the problem of network selection within the context of Digital Marketplace.

The network selection mechanism advocated by the Digital Marketplace lacked extensive and rigorous economic analysis. With the game theoretic analysis presented in this thesis, this deficiency has been addressed. As a result, should the Digital Marketplace be chosen as a management platform of wireless access networks of the future, the participating network operators can use the results derived in this thesis to formulate their pricing strategies in the most optimal way. The subscribers, on the other hand, are in the position to understand the prices they will be required to pay depending on their preferences for the requested service (in terms of the price weight) and the reputation of the participating network operators. More specifically, in Chapter~\ref{cha:direct}, the equilibrium bidding strategies for three special/extreme cases were derived: 1) when only reputation ratings of the network operators decide on the winning network operator, 2) when only the monetary bids of the network operators matter in the selection of the winner, and finally, 3) when all network operators are characterised by the same reputation rating. Furthermore, the equilibrium bidding strategies for only two network operators were analytically derived. However, they were shown to be suboptimal as they allow the network operators to submit a negative monetary bid.

In Chapter~\ref{cha:indirect}, by mathematically transforming the problem into an alternate form, it was shown that the equilibrium bidding strategies exist and are unique. Furthermore, they were explicitly derived in the case of two network operators and their costs assumed to be uniformly distributed. In that case, the expected prices the subscriber will have to pay for different values of the price weight were examined. Finally, four numerical methods for numerically approximating the equilibrium bidding strategies in the case of more than two network operators were proposed.

Finally, Chapter~\ref{cha:approximation} explored whether an auction format represented by the network selection mechanism employed in the Digital Marketplace can be modelled as an auction with common prior. In an auction with common prior, the range the costs can vary is the same for each bidder. In the first instance, the assumptions governing an auction with common prior were described, and the existence and uniqueness of the equilibrium bidding strategies were formally defined. Following that a numerical method tailored specifically to the auction with common prior was presented. Having derived the numerical method for approximating the equilibrium in the auction with common prior, the methodology for casting the original problem into the auction with common prior was discussed. Finally, the methodology for quantifying the accuracy of the approximation was presented, and the chapter concluded with the presentation of approximation results for four bidding scenarios with two, three, four and five bidders respectively.

It should be noted, however, that since the analysis documented in this thesis was kept as generic as possible, the results herein presented are easily extrapolated from the context of the Digital Marketplace.

Furthermore, an indirect contribution of this thesis was development of a numerical algorithm (EFSM) for approximating first-price sealed-bid auction with asymmetric bidders to the problem posed by the network selection mechanism employed by the Digital Marketplace. The EFSM numerical method tackles the unusual aspect of the said bidding problem.
% section conclusions (end)

\section{Further Work} % (fold)
\label{sec:further_work_conclusions}
Further work goes here.
% section further_work (end)
