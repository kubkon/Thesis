%!TEX root = /Users/jakubkonka/Thesis/Thesis.tex
\chapter{Digital Marketplace} % (fold)
\label{cha:dmp}

\minitoc
\vspace{10mm}

\section{Principles of Operation} % (fold)
\label{sec:principles_of_operation_dmp}
The Digital Marketplace (DMP) is a market-based framework for trading wireless communications services. It was developed with the heterogeneous wireless communications environment in mind, where the end-users (of communications services) have the ability to select a network operator that reflects their preferences on a per-service basis.

In its simplest form, there are three main groups of economic agents involved in the operation of the DMP: \emph{subscribers}, \emph{network operators}, and \emph{market provider}. This is depicted in Figure~\ref{fig:dmp_model_dmp}. The subscribers are the end-users of the communications services, and they act as the buyers in the DMP. The network operators, on the other hand, act as the sellers/bidders, and are responsible for providing the subscribers with services and facilitating network resources required to transport said services. In networking terminology, network operators are equivalent to mobile network operators (MNOs); for example, O2 or Vodafone in the UK. Lastly, the market provider is tasked with operating the DMP; thus providing common platform for all agents involved. It is left open-ended who should be the market provider; however, one of the following three choices is the most likely: a regulatory body, a consortium of network operators, or a single network operator on behalf of the regulatory body \cite{DMLeBodic00,DMIrvine02}.

\begin{figure}[p]
	\includegraphics[width=4in]{DMP/Figures/dmp_model}
	\caption{The business model of Digital Marketplace (adapted from~\cite{DMIrvine02})}
	\label{fig:dmp_model_dmp}
	\vspace{10mm}
	\includegraphics[width=3.5in]{DMP/Figures/no_decoupled}
	\caption{Decoupling of network operator into service and network providers (adapted from~\cite{DMIrvine02})}
	\label{fig:no_decoupled_dmp}
\end{figure}

It should be noted that the original specification of the DMP found for example in~\cite{DMLeBodic00,DMIrvine02,LeBodicThesis} differentiates between service and network providers. Thus, the network operator as an entity is decoupled into service and network providers as shown in Figure~\ref{fig:no_decoupled_dmp}. According to this model, the service provider is responsible for providing communications services to the end-user, while network provider facilitates (physical) network resources required to transport said services. A good example of a service provider is that of a mobile virtual network operator (MVNO). An MVNO, such as Giffgaff in the UK, provides services to the end-users, but does not necessarily own a physical network infrastructure; instead, they enter into a contract with an MNO, and use their network to transport users' services. For example, in the UK, Giffgaff has a contractual agreement with O2.

While the original specification of the DMP advocates decoupling of service and network provisions, in the research reported in this thesis we concentrate on the simplest business model; i.e., service and network provision is handled by one entity, a network operator. Nevertheless, all of the results presented in this thesis are applicable to the decoupled case. Indeed, it is a matter of replacing network operator with network provider; or replacing subscriber with service provider, and network operator with network provider. In the latter case, the negotiation process (or network selection mechanism which is described in Section~\ref{sec:network_selection_mechanism_dmp}) is between service and network providers. In this case, service provider acts as a buyer, while network providers are the sellers.

The next section provides a conceptual overview of the negotation process between the subscriber and the network operators. This process is termed \emph{network selection mechanism}, and is the means for the subscriber to select the network operators who reflects their service preferences best.
% section principles_of_operation_dmp (end)

\section{Network Selection Mechanism} % (fold)
\label{sec:network_selection_mechanism_dmp}
This section is organised as follows. Firstly, a high-level overview of auction theory is given, which is necessary to understand the fundamental assumptions governing the network selection mechanism in the DMP. Then, a conceptual overview of the network selection mechanism is provided.

\subsection{Primer in Auction Theory} % (fold)
\label{sub:primer_in_auction_theory_dmp}
As argued by many economists~\cite{Klemperer1999,Milgrom2004,Krishna10}, auction theory is one of the most prominent branches of economics. Examples of auctions being used in real life abounds: purchasing rare items of considerable value such as paintings, buying a house, or simply shopping on eBay. However, auctions are not only restricted to buying goods; they can also be used as a selling mechanism. Indeed, a perfect example of an auction used to sell goods is the well-known case of spectrum auctions which were used by both the US and the UK governments to sell the radio spectrum licenses to network operators. It is the purpose of this section to provide a high-level overview of the most important models and assumptions of auction theory.

\subsubsection{Common Auction Formats} % (fold)
\label{ssub:common_auction_formats_dmp}
There are many different auction formats reported in the auction theory literature; however, four are particularly popular, and are explored in this section. Those are \emph{English} (also known as open ascending-price), \emph{Dutch} (also known as open descending-price), \emph{first-price sealed-bid}, and \emph{second-price sealed-bid} auctions.

In an English auction, the person who conducts the sale, i.e., the auctioneer, calls out bids in an increasing fashion until there is only one interested bidder left. For example, the sale of a painting (or other work of art) would traditionally be facilitated by the mechanism of an English auction. In such an auction, the auctioneer would set the base price for the object to be sold, $x$ pounds say. Suppose further that some bidder A registered their interest (by raising their hand, or otherwise) in obtaining the object for the price of $x$. Then, bidder A would be proclaimed the highest bidder, and the auctioneer would call out another price, $y$, such that it is higher from the previous one, $x < y$ (hence, the alternative name of ascending-price auction). If other bidder, bidder B, say, registered their interest in obtaining the object for the new price of $y$, then bidder B would become the highest bidder, and the auctioneer would further increase the price. And so on, until no further interest was observed. The object would then go to the current highest bidder.

Similarly to an English auction, in a Dutch auction, the prices of the object for sale change in a sequential manner; however, in a Dutch auction, the price is decreasing. The auction is conducted in the following way. The auctioneer starts at a price $x$ say. If no bidder registers interest within the given time limit set by the auctioneer, then the price is decreased to $y$, say, such that $y < x$. And so on, until a bidder registers interests. Then, the object is sold for that price to that bidder. It is worth noting that English and Dutch auctions are an example of open auctions since every bidder observes the bids of all the other bidders. This is in contrast to first-price and second-price sealed-bid auctions which are discussed next.
% subsection common_auction_formats_dmp (end)

\subsubsection{Bidder Valuations} % (fold)
\label{ssub:bidder_valuations_dmp}

% subsection bidder_valuations_dmp (end)

\subsubsection{Standard versus Procurement Auctions} % (fold)
\label{ssub:standard_versus_procurement_auctions_dmp}
It is important to realise that procurement auctions are equivalent to standard auctions in the same setting~\cite{Krishna10}. Therefore, the abundance of results on standard auctions applies to procurement auctions with only certain small conceptual differences; for example, in standard auctions we talk about the maximum bid, while in procurement auctions about the minimum bid. We make use of this fact in this thesis, and provide proofs of only results not already covered in the literature on auctions in general, since if the result is proved in one case (be it either for standard or procurement auctions), it can immediately be adapted to the other case.
% subsection standard_versus_procurement_auctions_dmp (end)
% subsection primer_in_auction_theory_dmp (end)

\subsection{Conceptual Overview} % (fold)
\label{sub:conceptual_overview_dmp}
The process of negotiation (or the network selection mechanism) in the DMP is based on a procurement first-price sealed-bid auction. Unlike in a standard procurement first-price sealed-bid auction, the winning bid is a weighted (convex) combination of both the network operator's monetary bid and their reputation rating; we will refer to it as the \emph{compound bid}. The network operator is elected as the winner of the auction if their compound bid is the lowest in value, and accrues their monetary bid minus the cost of supporting the service. The monetary bid is equivalent to the price of supporting the service by the network operator. The precise definition of the price is left open-ended; one possibility, for example, would be to charge the buyer per unit of bandwidth. The weights in the compound bid are set by the subscriber before each auction, and are announced to the network operators. This effectively gives the subscriber the freedom to choose any combination ranging from: a low price for the service but also poor quality; to a high quality but for a high price \cite{DMLeBodic00}.

It is important to note that, out of sequential-bid and sealed-bid auctions, a procurement first-price sealed-bid auction was chosen due to the following reasons. Firstly, given the timing constraints in the DMP (e.g., the waiting time of the subscriber for the service to be admitted), and the difficulty in predicting the number of bids placed until the winner is selected in a sequential-bid auction, sealed-bid auctions were deemed as the most appropriate~\cite{DMLeBodic00}. Secondly, the rules governing a second-price sealed-bid auction may appear as counter-intuitive to the subscriber; that is, as mentioned in the previous section, the lowest bid secures the auction but the price paid equals the second-lowest bid. Lastly, since the subscribers not only base their network selection strategy on the offered price, but also on reputation, a first-price sealed-bid auction is the best fit to such a requirement.

Furthermore, since the communications services are traded on an individual service level, it might be difficult for the subscriber to judge the overall quality of the services supplied by a particular network operator \cite{DMIrvine02}. Therefore, one of the fundamental assumptions governing the operation of the DMP is that, by registering in the DMP, network operators agree to report on their contract fulfillments to the market provider; that is, they agree to report a binary value denoting the success in delivering the service to the subscriber within the agreed Quality of Service (QoS) bounds \cite{DMLeBodic00}. The value of $0$ denotes a failure, while the value of $1$ a success. The latest $d$ ($d>1$) reports are then used to compute the reputation rating of the network operator which will be used when a new service request arrives in the marketplace. Hence, assuming network operator $i$ admitted $t$ service requests, the formula for computing a reputation rating update is as follows (cf.~Section~3.2 in~\cite{DMLeBodic00})
\begin{equation}
    \label{eq:reputation_rating_update_dmp}
    r_i^{t+1} = \sum_{k = t-d}^d \frac{1 - report_i^k}{d},
\end{equation}
where $report_i^k$ denotes the $k^{\text{th}}$ binary report of the network operator $i$. Note that Equation~\eqref{eq:reputation_rating_update_dmp} implies $r_i^{t+1} = 0$ if the network operator $i$ has successfully delivered $d$ services to the subscriber, while $r_i^{t+1} = 1$ if has failed in all $d$ attempts. Furthermore, Equation~\eqref{eq:reputation_rating_update_dmp} implies that if the operator is consistently unreliable, their performance is reflected accordingly by their reputation rating history. Whilst, similarly, one failure in delivering the service does not immediately render a network operator unreliable; rather, it marginally affects their updated reputation rating. At the same time, at the end of each contract, the subscriber may report on their satisfaction (or Quality of Experience, QoE) with the service, for example, by submitting a mean opinion score in case of real-time services, and achieved throughput for non-real-time ones. The reputation rating update formula in Equation~\eqref{eq:reputation_rating_update_dmp} could then be modified to incorporate QoE, for instance, by taking an appriopriately weighted composition of both network operator's and subscriber's reports. Since this paper just barely scratches the surface of the reputation rating system maintained by the DMP, and the concept of QoE, the Readers are referred to \cite{DMLeBodic00, LeBodicThesis, DMIrvine02, DMMathur02, DMIrvine01, DMMcDiarmid06} for a more in-depth treatment of the former and to~\cite{Kilkki2008, BrooksHestnes2010, Fiedler2010, Shaikh2010} of the latter.
% subsection conceptual_overview_dmp (end)
% section network_selection_mechanism_dmp (end)

\section{Contributions of This Research to Digital Marketplace} % (fold)
\label{sec:contributions_of_this_research_to_digital_marketplace_dmp}

% section contributions_of_this_research_to_digital_marketplace_dmp (end)

\section{Summary} % (fold)
\label{sec:summary_dmp}

% section summary_dmp (end)
% chapter digital_marketplace_dmp (end)