%!TEX root = /Users/jakubkonka/Thesis/Thesis.tex
\chapter{Numerical Solution to the Bidding Problem in the Digital Marketplace}
\label{cha:numerical_solution_to_the_bidding_problem_in_the_digital_marketplace}

\minitoc
\vspace{10mm}

Something...\cite{Katzwer2012}

\section{Bidding Problem Revisited}
\label{sec:bidding_problem_revisited_numerical}
Recall the utility function of each network operator $i\in N$ in the indirect approach
\begin{equation*}
  u_i(\hat{b},\hat{c}) = \left\{
  \begin{array}{l l}
    \displaystyle\frac{1}{w}\left(\hat{b}_i-\hat{c}_i\right) & \;\text{if } \hat{b}_i < \displaystyle\min_{j\neq i}\hat{b}_j,\\[2ex]
    0 & \;\text{if } \hat{b}_i > \displaystyle\min_{j\neq i}\hat{b}_j,
  \end{array}\right.
\end{equation*}
where $w\neq 0$, and
\begin{equation*}
  \hat{b}_i = wb_i + (1-w)r_i, \quad\text{and}\quad \hat{c}_i = wc_i + (1-w)r_i.
\end{equation*}
In equilibrium, the bids of each network operator equal $\hat{b_i} = \hat{b}_i(\hat{c}_i)$, where $\hat{b}_i$ is the equilibrium bidding function. Denote by $\hat{c}_i(\hat{b}_i)\equiv \hat{b}_i^{-1}(\hat{b}_i)$ an inverse equilibrium bidding function for each network operator $i\in N$. Therefore, the expected utility for each network operator $i\in N$ can be written as
\begin{align*}
  \Pi_i(\hat{b}_i,\hat{c}_i,\hat{b}_{-i},\hat{c}_{-i})
  &\equiv (\hat{b}_i - \hat{c}_i)P\{\text{winning}\mid\hat{b}_i\}\\
  &= (\hat{b}_i - \hat{c}_i)Q_i(\hat{b}_i),
\end{align*}
where
\begin{equation*}
Q_i(\hat{b}_i) = \prod_{j\neq i}\left( 1 - F_j(\hat{c}_j(\hat{b}_i)) \right)
\end{equation*}
is the probability that network operator $i$ is the lowest bidder, and $F_j$ is the distribution function of $\hat{c}_j$.

The first order condition for maximizing network operator $i$'s expected utility is
\begin{equation}
  \label{eq:foc_numerical}
  \frac{d}{d\hat{b}_i}\Pi_i(\hat{b}_i,\hat{c}_i,\hat{b}_{-i},\hat{c}_{-i}) = Q_i(\hat{b}_i) + (\hat{b}_i - \hat{c}_i)\cdot\frac{d}{d\hat{b}_i}Q_i(\hat{b}_i) = 0,
\end{equation}
where
\begin{equation*}
  \frac{d}{d\hat{b}_i}Q_i(\hat{b}_i) = (-1)\sum_{j\neq i} f_j(\hat{c}_j(\hat{b}_i))\frac{d}{d\hat{b}_i}\hat{c}_j(\hat{b}_i)\prod_{k\neq j} \left( 1 - F_k(\hat{c}_k(\hat{b}_i)) \right),
\end{equation*}
and $f_j\equiv\frac{d}{dx}F_j$ (where $x$ is a dummy variable) is the density function of $\hat{c}_j$.

Noting that in equilibrium $\hat{c}_i = \hat{c}_i(\hat{b}_i)$, letting $\hat{b}_i = b$, and rearranging terms in Equation~\eqref{eq:foc_numerical} yields
\begin{align}
  \label{eq:foc_simplified_numerical}
  \frac{1}{b - \hat{c}_i(b)} 
  &= \frac{\sum_{j\neq i} f_j(\hat{c}_j(b))\frac{d}{db}\hat{c}_j(b)\prod_{k\neq j} \left( 1 - F_k(\hat{c}_k(b)) \right)}{\prod_{j\neq i} \left( 1 - F_j(\hat{c}_j(b)) \right)}\nonumber \\[2ex]
  &= \sum_{j\neq i}\frac{f_j(\hat{c}_j(b))}{1 - F_j(\hat{c}_j(b))}\cdot\frac{d}{db}\hat{c}_j(b).
\end{align}
Summing Equation~\eqref{eq:foc_simplified_numerical} over all $n$ network operators yields
\begin{equation}
  \label{eq:foc_summed_numerical}
  \frac{1}{n-1}\sum_{i=1}^n \frac{1}{b - \hat{c}_i(b)} = \sum_{i=1}^n \frac{f_i(\hat{c}_i(b))}{1 - F_i(\hat{c}_i(b))}\cdot\frac{d}{db}\hat{c}_i(b).
\end{equation}
Subtracting Equation~\eqref{eq:foc_simplified_numerical} from \eqref{eq:foc_summed_numerical} yields 
\begin{equation*}
  \frac{1}{n-1}\sum_{i=1}^n \frac{1}{b - \hat{c}_i(b)} - \frac{1}{b - \hat{c}_i(b)} = \frac{f_i(\hat{c}_i(b))}{1 - F_i(\hat{c}_i(b))}\cdot\frac{d}{db}\hat{c}_i(b)
\end{equation*}
which leads to the system of nonlinear ordinary differential equations (ODE)
\begin{equation}
  \label{eq:foc_ode_numerical}
  \frac{d}{db}\hat{c}_i(b) = \frac{1 - F_i(\hat{c}_i(b))}{f_i(\hat{c}_i(b))}\left[ \frac{1}{n-1}\sum_{i=1}^n \frac{1}{b-\hat{c}_i(b)} - \frac{1}{b-\hat{c}_i(b)} \right]
\end{equation}
for $i=1,2,\dotsc,n$.

\section{Summary}
\label{sec:summary_numerical}

