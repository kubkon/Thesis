%!TEX root = /Users/jakubkonka/Thesis/Thesis.tex
\chapter{Casting Network Selection Mechanism into Common Prior Setting}
\label{cha:approximation}
\annotate{C6.3}{This chapter presents a methodology for approximating the DMP network selection mechanism with an asymmetric FPA auction with common prior. It is further argued that this methodology constitutes a possible resolution to the potential problem of numerical instability of the FSM and EFSM methods.}

\annotate{C6.4}{Fibich and Gavish~\cite{FibichGavish2011} showed that the FSM method and its derivatives, such as the EFSM method, become numerically unstable for large numbers of bidders. The issue has not impacted the results presented in this thesis thus far due to the fact that only the scenarios with as many as 4 network operators were considered. However, it is important to acknowledge the fact that the issue exists and, sooner or later, for large number of network operators, it will affect the numerical solutions generated by FSM and, more importantly, EFSM methods. Therefore, it is vital to address the issue on a proactive rather than reactive basis.}

\annotate{}{The most obvious way of addressing the issue would be to employ a different numerical method in place of the EFSM method. However, to the best of author's knowledge, the EFSM method is the only numerical algorithm in existence that considers all nontrivial equilibria to the system of ODEs in Equation~\eqref{eq:foc_ode_indirect} with lower and upper boundary conditions in Equations~\eqref{eq:foc_ode_lower_boundary_indirect}~and~\eqref{eq:foc_ode_upper_boundary_indirect} respectively. Furthermore, it is not immediately obivious how the EFSM method would have to be modified to be based entirely on methods that are not FSM derivative, and hence, do not possess numerical instability issues.}

\annotate{}{In this chapter, an alternative approach is presented. It is explored whether an auction format represented by the DMP network selection mechanism can be modelled as an asymmetric FPA auction with common prior (henceforth, referred to as CP auction). In a CP auction, the range the costs can vary is the same for each bidder. More formally, the cost distributions for each bidder share the same support. By modelling the DMP network selection mechanism as a CP auction, the numerical solution methods (other than the FSM-based methods) presented in Hubbard and Parsch~\cite{HubbardPaarsch2011}, and extensively studied by the economic community, could be used to approximate the solution to the DMP auction. This would allow network operators to consider a simpler bidding problem for which there are many well-defined numerical solutions. As a result, presented with a DMP auction, network operators could bid according to the equilibrium bidding strategies of the corresponding CP auction while approximately retaining the expected utility if bidding according to the equilibrium bidding strategies of the DMP auction, and hence, avoiding the need to use the EFSM method to solve the DMP auction.}

\annotate{C6.6}{Modelling of the DMP auction as a CP auction assumes that the network operators will use the equilibrium bidding strategies of the CP auction (CP strategies) as bidding strategies in the DMP auction. However, by Proposition~\ref{prop:characterization_of_the_equilibrium_indirect}, the CP strategies do not constitute an equilibrium to the DMP auction; they are merely used as \emph{approximations} to the actual equilibrium bidding strategies of the DMP auction (equilibrium strategies). Therefore, there exists possibility that a network operator might exploit this fact by bidding according to the equilibrium strategies while other network operators will bid according to the CP strategies. Concurrently, however, since the equilibrium strategies can only be derived using the EFSM method, it is likely that the derivation might fail due to the numerical instability of the algorithm. All in all, each network operator faces a tradeoff: bid according to the equilibrium strategies but risk lack of convergence, or bid according to CP strategies but risk other network operators bidding according to the equilibrium strategies. Of course, the magnitude of the problem decreases dramatically as the number of network operators involved in the DMP increases. For then the numerical instability will lead to the divergence of the EFSM algorithm and render the derivation of the equilibrium strategies impossible. Hence, the network operators will be forced to rely on the CP strategies.}

The analysis is organised as follows. In the first instance, the assumptions governing the CP auction are described, and the existence and uniqueness of the equilibrium bidding strategies is formally defined. Following that the FSM numerical method tailored specifically to the CP auction setting is presented. It is worth noting at this point that the CP version of the FSM algorithm corresponds to the original FSM algorithm first presented by Bajari~\cite{Bajari2001a} (cf.~Algorithm 1 in \cite{Bajari2001a}). Having derived the numerical method for approximating the equilibrium in the CP auction setting, the methodology for casting the DMP bidding scenario into a CP auction setting is discussed. That is, it is showed how a DMP auction can be approximated as a CP equivalent. Furthermore, the methodology for quantifying the accuracy of the approximation is presented. Finally, the chapter concludes with the presentation of approximation results for four bidding scenarios with two, three, four and five bidders respectively.

\section{Mathematical Description} % (fold)
\label{sec:mathematical_description_approximation}

Following the notation of Chapter~\ref{cha:indirect}, let each bidder $i$ be characterised by the utility function
\begin{equation}
  \label{eq:utility_approximation}
    u_i(b,c) = \left\{
  \begin{array}{l l}
    b_i-c_i & \;\text{if } b_i < \displaystyle\min_{j\neq i}b_j,\\
    0 & \;\text{if } b_i > \displaystyle\min_{j\neq i}b_j,
  \end{array}\right.
\end{equation}
where, as before, $b = (b_1,\ldots,b_n)$, and $c = (c_1,\ldots,c_n)$. In the CP auction, it is assumed that each bidder $i$ draws their cost from common support across all bidders; i.e., let
\begin{equation}
  c_i\in [\underline{c}, \bar{c}] \quad\text{for all } i\in N \text{ such that } [\underline{c}, \bar{c}]\subseteq [0, 1].
\end{equation}

Let $F_i$ be the distribution function of $c_i$ for all $i\in N$. Note that the distribution functions between bidders need not be equal, and hence, the problem is that of an asymmetric FPA.

It is further assumed that
\begin{assumptions}
\label{ass:assumptions_common_priors_approximation}
Assume that
\begin{enumerate}
  \item $F_i$ is differentiable over $(\underline{c}, \bar{c}]$ with a derivative $f_i$ locally bounded away from zero over this interval;
  \item $F_i$ is atomless; and
  \item $F_i(c)>0$ for all $c\in [\underline{c}, \bar{c}]$ and $i\in N$.
\end{enumerate}
\end{assumptions}
These assumptions correspond to Assumptions~A.1 and Theorem~U.1 in Lebrun~\cite{Lebrun2006}, and, as shown by Lebrun, with these assumptions satisfied, there exists one and only one pure-strategy Bayesian Nash equilibrium where bidders engage in serious bidding; that is, bid at least their cost. Formally,
\begin{proposition}[Characterisation of the Equilibrium in Common Prior Setting]
\label{prop:characterization_of_the_equilibrium_in_common_priors_setting_approximation}
Let Assumptions~\ref{ass:assumptions_common_priors_approximation} be satisfied. There exists one and only one pure-strategy Bayesian Nash equilibrium where bidders submit at least their costs. In every such equilibrium, bidder $i\in N$ follows a bid function $b_i$, for all $1\leq i\leq n$ such that its inverse, $c_i= b_i^{-1}$, satisfy the following system of differential equations
\begin{equation}
  \frac{d}{db}c_i(b) = \frac{1 - F_i(c_i(b))}{f_i(c_i(b))}\left[ \frac{1}{n-1}\sum_{k=1}^n \frac{1}{b-c_k(b)} - \frac{1}{b-c_i(b)} \right]
\end{equation}
for all $1\leq i\leq n$, with the following lower boundary condition
\begin{equation}
  \label{eq:foc_ode_lower_boundary_approximation}
  c_i(\underline{b}) = \underline{c}
\end{equation}
and the upper boundary condition
\begin{equation}
  \label{eq:foc_ode_upper_boundary_approximation}
  c_i(\bar{c}) = \bar{c}
\end{equation}
for all $1\leq i\leq n$.
\end{proposition}

In effect, Proposition~\ref{prop:characterization_of_the_equilibrium_in_common_priors_setting_approximation} is a special case of Proposition~\ref{prop:characterization_of_the_equilibrium_indirect}. That is, the equilibrium bidding functions still have to satisfy the system of nonlinear ODEs given by Equation~\eqref{eq:foc_ode_indirect}; however, in this case, the lower boundary condition reduces to
\begin{equation}
  c_i(\underline{b}) = \underline{c},
\end{equation}
and the upper boundary condition to
\begin{equation}
  c_i(\bar{c}) = \bar{c},
\end{equation}
i.e., the bids never exceed the upper extremity of the common support range.

It should be noted that, even though the bidding problem is considerably simpler than the original one discussed in this thesis (cf.~Chapter~\ref{cha:indirect}), it still involves finding the lower bound on bids, and hence, the closed-form solution exists only in a handful of special cases~\cite{Krishna10,HubbardPaarsch2011}. However, as presented by Hubbard and Paarsch~\cite{HubbardPaarsch2011}, the problem can be approximated using numerical methods, which is discussed in the next section.
% section mathematical_description (end)

\section{Numerical Solutions} % (fold)
\label{sec:numerical_solutions}
In this section, a CP auction is approximated using the FSM method already introduced in Section~\ref{sec:numerical_analysis_indirect}, Chapter~\ref{cha:indirect}, but tailored to the problem at hand. The FSM method was chosen due to its relatively low implementation complexity (compared to the PPM method), and the fact that it was also used to approximate the DMP bidding problem. Therefore, in terms of the numerical accuracy and stability, the numerical solutions to the DMP and CP auctions should be of comparable quality. Furthermore, since the discussion concentrates on a relatively small number of bidders, the FSM (and the EFSM) method is still well-behaved numerically.

\subsection{Forward Shooting Method} % (fold)
\label{sub:forward_shooting_method}
To briefly recap, the FSM method was first proposed by Bajari~\cite{Bajari2001a} (cf.~Algorithm~1 in \cite{Bajari2001a}). The method aims at finding the best approximation of the lower bound on bids, $\underline{b}$, by successively picking a value from the feasible interval, $(\underline{c}, \bar{c})$, and verifying whether a numerical solution to the initial value problem
\begin{equation}
  \label{eq:fsm_initial_value_problem_approximation}
  \begin{array}{ll}
     \displaystyle\frac{d}{db}c_i(b) &= \displaystyle\frac{1 - F_i(c_i(b))}{f_i(c_i(b))}\left[ \frac{1}{n-1}\sum_{k=1}^n \frac{1}{b-c_k(b)} - \frac{1}{b-c_i(b)} \right]\\[2ex]
    c_i(\underline{b}) &= \underline{c}
  \end{array}
\end{equation}
for all $i\in N$ satisfies the following three conditions: 1) it is a function mapping $[\underline{b}, \bar{c}]$ into $[\underline{c}, \bar{c}]$, that is,
\begin{equation}
  \label{eq:fsm_condition_1_approximation}
    s: [\underline{b}, \bar{c}]\to [\underline{c}, \bar{c}];
\end{equation}
2) it is monotonically increasing everywhere except possibly at $\underline{c}$, that is,
\begin{equation}
  \label{eq:fsm_condition_2_approximation}
  b_1 < b_2\implies s(b_1) < s(b_2) \text{ for all }b_1,b_2\in [\underline{b}, \bar{c});
\end{equation}
and 3) each function value is strictly lower than its argument except possibly at $\bar{c}$, that is,
\begin{equation}
  \label{eq:fsm_condition_3_approximation}
  s(b) < b \text{ for all }b\in [\underline{b}, \bar{c}).
\end{equation}

The pseudo-code for the FSM is depicted in listing Algorithm~\ref{alg:forward_shooting_method_approximation}. Note that the algorithm is almost identical to the FSM version tailored to the DMP auction (cf.~Algorithm~\ref{alg:forward_shooting_method_indirect}), with only differences being the definition of the set of permissible functions, $S$, and the algorithm's search region delimited by $low$ and $high$ variables. Hence, the discussion of the algorithm is omitted, and the reader is referred to Section~\ref{sub:forward_shooting_method_indirect}.

\begin{algorithm}
\caption{Forward shooting method (common prior version; Bajari~\cite{Bajari2001a})}
\label{alg:forward_shooting_method_approximation}
\begin{algorithmic}[1]
\Require{$\epsilon\in (0, \bar{c} - \underline{c}); low, high\in [\underline{c}, \bar{c}]$ such that $low\leq high$}
\Ensure{Approximation to $\underline{b}$}
  \Statex
  \Let{$low$}{$\underline{c}$}
  \Let{$high$}{$\bar{c}$}
  \Statex
  \While{$high-low > \epsilon$}
    \Let{$guess$}{$0.5\cdot(low + high)$}
    \Let{$bids$}{$[guess, \bar{c})$}
    \Let{$(costs_1,\dotsc,costs_n)$}{solve~\eqref{eq:fsm_initial_value_problem_approximation} with initial value $\underline{b} = guess$}
    \StatexIndent[7.5]{evaluated at points $b\in bids$}
    \If{$(bids,costs_i)$ satisfies~\eqref{eq:fsm_condition_1_approximation}, \eqref{eq:fsm_condition_2_approximation} and \eqref{eq:fsm_condition_3_approximation} for all $i\gets 1$ to $n$}
      \Let{$high$}{$guess$}
    \Else
      \Let{$low$}{$guess$}
    \EndIf
  \EndWhile
  \Statex
  \Let{$\underline{b}$}{$0.5\cdot(low + high)$}
\end{algorithmic}
\end{algorithm}

Similarly to the implementation of the FSM (and EFSM) algorithm for the DMP auction, the approximation results presented in this chapter have been derived using the GSL implementation of the Embedded Runge-Kutta-Fehlberg (4,5) method.

% subsection forward_shooting_method (end)

\subsection{Verification} % (fold)
\label{sub:verification}
Before proceeding with the modelling and analysis, the FSM algorithm was tested for correct implementation. The bidding scenario used to verify the algorithm is taken from the Bajari's paper \cite{Bajari2001a}. There are three bidders, and each is characterised by a truncated normal distribution but with different mean and standard deviation parameters (see Table~\ref{tab:verification_approximation}). Furthermore, each bidder draws their cost from common costs' range, $c_i\in [2,8]$.

Figure~\ref{fig:fsm_common_priors_verification_approximation} depicts the numerically approximated solution to the problem. It is clear that the approximation agrees with that of Bajari's~\cite{Bajari2001a} (cf.~Figure~1 in \cite{Bajari2001a}). Furthermore, in Figure~\ref{fig:fsm_common_priors_verification_sufficiency_approximation}, the numerical solution is verified whether it satisfies the sufficiency condition for an equilibrium; that is, whether the numerically derived bidding strategy for each bidder is a best response to the bidding strategies of the remaining bidders. As expected, the solution satisfies the sufficiency condition, and hence, it is concluded that the algorithm was implemented correctly.

\begin{table}[t]
  \caption{Test bidding scenario}
  \vspace{0.5cm}
  \begin{tabular*}{0.5\columnwidth}[L]{@{\extracolsep{\fill}}r c c}
    \hlx{vhv}
    & \textbf{Mean}, $\mu_i$ & \textbf{Standard deviation}, $\sigma_i$\\
    \hlx{vhv}
    \textbf{Bidder 1} & $4$ & $1.5$\\
    \textbf{Bidder 2} & $5$ & $1.5$\\
    \textbf{Bidder 3} & $6$ & $1.5$\\
    \hlx{vhs}
  \end{tabular*}
  \label{tab:verification_approximation}
\end{table}

\begin{figure}[p!]
  \includegraphics[width=\figsize]{Approximation/Figures/fsm_common_priors_verification}
  \caption{FSM solution to the test common prior bidding problem}
  \label{fig:fsm_common_priors_verification_approximation}
  \vspace{10mm}
  \includegraphics[width=\figsize]{Approximation/Figures/fsm_common_priors_verification_sufficiency}
  \caption{FSM solution satisfies sufficiency condition for an equilibrium}
  \label{fig:fsm_common_priors_verification_sufficiency_approximation}
\end{figure}

% subsection verification (end)

In this section, version of the FSM algorithm tailored to the CP auction was presented, and verified for correct implementation. The next section explores the methodology for approximating a DMP auction with a CP auction.
% section numerical_solutions (end)

\section{Network Selection Mechanism Cast into Common Prior Setting} % (fold)
\label{sec:network_selection_mechanism_cast_into_common_priors_setting_approximation}
In this section, the DMP auction is firstly modelled as a CP auction where bidders are characterised by costs distributed according to a truncated normal distribution. Then, the methodology used to quantify the accuracy of approximations is outlined.

\begin{figure}[p!]
  \includegraphics[width=\figsize]{Approximation/Figures/dmp_to_common_priors}
  \caption{Mapping probability distributions from the DMP auction into truncated normal distributions with common support}
  \label{fig:dmp_to_common_priors_approximation}
\end{figure}

\subsection{Modelling using Truncated Normal Distribution} % (fold)
\label{sub:modeling_using_truncated_normal_distribution_approximation}
Recall from Chapter~\ref{cha:indirect} that, in the DMP auction, each bidder $i$ draws their cost from a uniform distribution with the support
\begin{equation}
  [(1-w)r_i, (1-w)r_i + w] = [\underline{c}_i, \bar{c}_i] \subset [0,1].
\end{equation}
In order to simplify the exposition of the concepts presented in this chapter, the costs-hat, $\hat{c}_i$, introduced in Chapter~\ref{cha:indirect} will be referred to as costs, $c_i$. Therefore, in the general case, unless the bidders are characterised by the same reputation rating, that is $r_i=r_j$ for all $i,j\in N$, their distributions' supports will not overlap fully; i.e.,
\begin{equation}
  [\underline{c}_i,\bar{c}_i] \neq [\underline{c}_j,\bar{c}_j], \quad i\neq j \text{ and } i,j\in N.
\end{equation}

Recall further that, in a CP auction, every bidder is characterised by a distribution (of costs) with common support across all bidders. Hence, in order to model any DMP bidding scenario, firstly, it needs to be agreed on a support that is common to every bidder and, at the same time, encompasses the supports of every individual bidder from the original (DMP) auction. The smallest such support is
\begin{equation}
  \label{eq:domain_common_priors_approximation}
  [\underline{c},\bar{c}] = \displaystyle\left[\min_{i\in N}\{\underline{c}_i\}, \max_{i\in N}\{\bar{c}_i\}\right] \subset [0,1].
\end{equation}
To see this, recall that, for any given $w\in (0,1)$, assuming $r_1\leq\cdots\leq r_n$ with at least one inequality strict, it follows $\underline{c}_1\leq\cdots\leq\underline{c}_n$ and $\bar{c}_1\leq\cdots\leq\bar{c}_n$ with at least one inequality strict. Further let $C_i = [\underline{c}_i, \bar{c}_i]$; then $C = \bigcup_{i\in N} C_i$ is the smallest set containing all sets $C_i$ for all $i\in N$. Since $C_i$ is closed for all $i\in N$, it follows that $C$ is closed, and $C = [\underline{c}, \bar{c}]$ such that $\underline{c} \leq \underline{c}_i$ and $\bar{c}_i\leq \bar{c}$ for all $i\in N$, which is equivalent to $[\min_{i\in N}\{\underline{c}_i\}, \max_{i\in N}\{\bar{c}_i\}]$.

All that remains is to then select a family of distributions which captures the numerical ranges of the original supports as closely as possible. To provide an illustrative example, let there be 2 bidders such that $\underline{c}_1 < \underline{c}_2 < \bar{c}_1 < \bar{c}_2$. Each bidder is characterised by a uniform distribution. \annotate{C6.5}{The common support in this case equals $[\underline{c}, \bar{c}] = [\underline{c}_1, \bar{c}_2]$. Firstly, recall that the chosen distirbutions have to satisfy Assumptions~\ref{ass:assumptions_common_priors_approximation}. Thus, uniform distributions considered over the common support cannot be chosen since they violate those assumptions. To see this, let $F_1$ be the cumulative distribution function (cdf) of the uniform distribution with the support $[\underline{c}_1, \bar{c}_1]$. Extended to the common support $[\underline{c}, \bar{c}]$, the derivative of $F_1$, the probability density function (pdf), is zero over the interval $[\bar{c}_1, \bar{c}] = [\bar{c}_1, \bar{c}_2]$, and hence, it is not locally bounded away from zero over the common support. As a result, it is necessary to choose distributions such that they satisfy Assumptions~\ref{ass:assumptions_common_priors_approximation}, and at the same time, possess the shape characteristics similar to the uniform distribution, such as symmetry about the mean.} One possible way of casting this scenario into common prior setting is to model the distributions of both bidders as truncated normal distributions truncated to the interval $[\underline{c}_1, \bar{c}_2]$, and with differing mean and standard deviation parameters. This is depicted in Figure~\ref{fig:dmp_to_common_priors_approximation}.

In order to describe the truncated normal distribution, firstly recall the pdf of standard normal distribution
\begin{equation}
  \label{eq:pdf_standard_normal_approximation}
  \displaystyle\phi(c) = \frac{1}{\sqrt{2\pi}} \exp\left\{-\frac{1}{2}c^2\right\},
\end{equation}
and cdf
\begin{equation}
  \label{eq:cdf_standard_normal_approximation}
  \displaystyle\Phi(c) = \int_{-\infty}^{c}\phi(c)dc = \frac{1}{2}\left[ 1 + \erf\left(\frac{c}{\sqrt{2}}\right) \right]
\end{equation}
for all $c\in\mathbb{R}$. The pdf of the truncated normal distribution, truncated to the interval $c\in[\underline{c},\bar{c}]$, can then be described in terms of the pdf of the standard normal distribution as follows
\begin{equation}
  \label{eq:pdf_truncated_normal_approximation}
  \displaystyle f(c; \mu, \sigma, \underline{c}, \bar{c}) = \frac{\frac{1}{\sigma}\phi\left(\frac{c-\mu}{\sigma}\right)}{\Phi\left(\frac{\bar{c}-\mu}{\sigma}\right) - \Phi\left(\frac{\underline{c}-\mu}{\sigma}\right)}
\end{equation}
where $\mu\in\mathbb{R}$ is the mean (or location) of the distribution, and $\sigma^2\geq 0$ is the variance (or squared scale)~\cite{JohnsonNormal1994,Cohen1991}. Similarly, the cdf of the truncated normal distribution can be defined as follows
\begin{equation}
  \label{eq:cdf_truncated_normal_approximation}
  \displaystyle F(c; \mu, \sigma, \underline{c}, \bar{c}) = \int_{-\infty}^{c}f(c;\mu,\sigma,\underline{c},\bar{c})dc
  = \frac{\Phi\left(\frac{c-\mu}{\sigma}\right) - \Phi\left(\frac{\underline{c}-\mu}{\sigma}\right)}{\Phi\left(\frac{\bar{c}-\mu}{\sigma}\right) - \Phi\left(\frac{\underline{c}-\mu}{\sigma}\right)}.
\end{equation}

Before moving on to discussing the methodology for quantifying the accuracy of the approximations, consider bidding scenario summarized in Table~\ref{tab:verification_indirect} in Chapter~\ref{cha:indirect}. Suppose this scenario was cast into common prior setting where bidders are characterised by truncated normal distributions. Firstly, it can be noted that the supports for both bidders are
\begin{equation}
  [\underline{c}_1, \bar{c}_1] = [0.125, 0.625]
\end{equation}
for bidder 1, and
\begin{equation}
  [\underline{c}_2, \bar{c}_2] = [0.375, 0.875]
\end{equation}
for bidder 2, while the common support is given by
\begin{equation}
  [\underline{c},\bar{c}] = [\underline{c}_1, \bar{c}_2] = [0.125, 0.875].
\end{equation}

\begin{figure}[t]
  \includegraphics[width=\figsize]{Approximation/Figures/modelling_params}
  \caption{Choosing parameters for the truncated normal distributions of the bidders}
  \label{fig:modelling_params_approximation}
\end{figure}

\annotate{C6.2}{Secondly, the distribution specific parameters (mean and standard deviation) need to be specified for each bidder. The choice of the parameters is motivated by the shape of the normal distribution. Therefore, the midpoints of the original supports are picked as means for both bidders, that is,
\begin{equation}
  \mu_i = \underline{c}_i + \frac{\bar{c}_i - \underline{c}_i}{2} = \underline{c}_i + \frac{w}{2}.
\end{equation}
Furthermore, noting that, in the case of normal distribution, 95\% of all the values falls within 2 standard deviations away from the mean~\cite{JohnsonNormal1994}, the standard deviations are selected to be equal to the quarter of the length of the original supports, that is,
\begin{equation}
  \sigma_i = \frac{\bar{c}_i - \underline{c}_i}{4} = \frac{w}{4}.
\end{equation}
In this way, for each bidder, 95\% of all the costs falls within the interval $[\underline{c}_i, \bar{c}_i]$, and therefore, the probability of drawing cost outside this interval is minimised. With this choice of parameters, the truncated normal distributions are effectively imitating uniform distributions with support $[\underline{c}_i, \bar{c}_i]$ for each bidder. This is depicted in Figure~\ref{fig:modelling_params_approximation} as the shaded region under the bell curve.}

\begin{table}[t]
  \caption{Numerical values of the chosen truncated normal distribution parameters}
  \vspace{0.5cm}
  \begin{tabular*}{0.5\columnwidth}[L]{@{\extracolsep{\fill}}r c c}
    \hlx{vhv}
    & \textbf{Mean}, $\mu_i$ & \textbf{Standard deviation}, $\sigma_i$\\
    \hlx{vhv}
    \textbf{Bidder 1} & $0.375$ & $0.125$\\
    \textbf{Bidder 2} & $0.625$ & $0.125$\\
    \hlx{vhs}
  \end{tabular*}
  \label{tab:test_truncated_normal_params_approximation}
\end{table}

\begin{figure}[p!]
  \includegraphics[width=\figsize]{Approximation/Figures/test_truncated_normal_pdfs}
  \caption{Pdfs of the truncated normal distributions from the CP bidding problem characterised by: $\underline{c}=\underline{c}_1=0.125$, $\bar{c}=\bar{c}_2=0.875$, and $\mu_1=0.375$ and $\sigma_1=0.125$ for bidder 1, and $\mu_2=0.625$ and $\sigma_2=0.125$ for bidder 2}
  \label{fig:test_truncated_normal_pdfs_approximation}
  \vspace{10mm}
  \includegraphics[width=\figsize]{Approximation/Figures/test_truncated_normal_bids}
  \caption{FSM solution to the CP bidding problem characterised by: $\underline{c}=\underline{c}_1=0.125$, $\bar{c}=\bar{c}_2=0.875$, and $\mu_1=0.375$ and $\sigma_1=0.125$ for bidder 1, and $\mu_2=0.625$ and $\sigma_2=0.125$ for bidder 2}
  \label{fig:test_truncated_normal_bids_approximation}
\end{figure}

Table~\ref{tab:test_truncated_normal_params_approximation} summaris es the numerical values of the described parameters, while the resultant pdfs are depicted in Figure~\ref{fig:test_truncated_normal_pdfs_approximation}, and Figure~\ref{fig:test_truncated_normal_bids_approximation} shows the resultant equilibrium bidding strategies for both bidders. It is worth noting that the pdfs match the illustrative example shown in Figure~\ref{fig:dmp_to_common_priors_approximation}. Furthermore, note that the pdfs for both bidders, as intended, are centred around the midpoints of their original supports respectively, and they tail off to zero as the bounds of the supports are reached.
% subsection modelling (end)

\subsection{Methodology for Quantifying Accuracy of the Approximations} % (fold)
\label{sub:methodology_for_quantifying_accuracy_of_the_approximations_approximation}
There are two fundamental questions that need to be addressed when it comes to quantifying accuracy of the approximations. First, how can the predictions (in terms of the equilibrium bidding strategies) produced by both auction types be compared, and second, how such a comparison can be quantified to allow for a programmatic treatment of the problem (thus, removing the possibility of human error when visually comparing the results). Two metrics will be considered: buyer's expected price, and \emph{ex ante} expected utility for each bidder. In this way, an indicator of how better off (or worse off) is the buyer and each of the bidders is obtained; that is, all agents involved in the auction are considered.

The buyer's expected price is equivalent to the expected value of the winning bid; that is,
\begin{equation}
  \label{eq:expected_price_approximation}
  p = E[b_i(c_i) \:\vert\: b_i(c_i) < \min_{j\neq i} b_j(c_j)],
\end{equation}
where $b_i$ is the equilibrium bidding function for all $i\in N$. Since an analytical derivation of the closed-form solution is not straightforward, similarly to the analysis presented in Section~\ref{sub:subscriber_s_perspective_expected_prices_indirect}, Chapter~\ref{cha:indirect}, the buyer's expected price is estimated numerically. That is, for each considered bidding scenario, the costs for each bidder are pseudo-randomly drawn from uniform distribution, the corresponding equilibrium bids are computed, and the minimum is chosen as the winning bid (price). This procedure is repeated 1000 times, yielding 1000 i.i.d.~observations of the price which are then averaged to give an estimate of the expected price (consequence of the Strong Law of Large Numbers; see Section~\ref{sub:strong_law_of_large_numbers_notation}, Appendix~\ref{cha:notation}).

In order to define the bidder's \emph{ex ante} expected utility, with some abuse of notation, the expected utility function for each bidder $i\in N$ as defined in Equation~\eqref{eq:def_expected_utility_indirect}, Chapter~\ref{cha:indirect} is restated here:
\begin{equation}
  \label{eq:expected_utility_approximation}
  \Pi_i(c_i) = (b_i(c_i) - c_i)\cdot \prod_{j\neq i} \left( 1 - F_j(b_j^{-1}(b_i(c_i))) \right)
\end{equation}
where $b_i$ is the equilibrium bidding function, and $F_i$ is the distribution function of costs for bidder $i$. The \emph{ex ante} expected utility is then equivalent to the expected value of the expected utility; that is,
\begin{equation}
  \label{eq:ex_ante_expected_utility_approximation}
  \Pi_i = E[\Pi_i(c_i)] = \int_{\underline{c}_i}^{\bar{c}_i} \Pi_i(t)dF_i(t)
\end{equation}
for all $i\in N$. In other words, the \emph{ex ante} expected utility can be thought of as the average expected utility for each bidder for each considered bidding scenario, and it follows from the definition of \emph{ex ante} expected payments in a standard first-price auction put forward by Krishna~\cite{Krishna10} (cf.~Section 2.4 Revenue Comparison in~\cite{Krishna10}).

The way the aforementioned metrics are actually computed deserves a more elaborate explanation. The numerical derivation of equilibrium in CP auction relies on approximating the bidders' distributions of costs with truncated normal distributions with common support, as discussed in Section~\ref{sub:modeling_using_truncated_normal_distribution_approximation}. When computing the expected price and \emph{ex ante} expected utilities for all bidders in the CP auction, it is assumed, however, that the bidders draw their costs from their actual (uniform) distributions but use the equilibrium bidding strategies derived for the CP auction with truncated normal distributions to compute their bids. In this way, when computing the expected price and \emph{ex ante} expected utilities, the bidders' distributions of costs are not misrepresented, and hence, ensure the comparison results of casting the DMP auction into CP auction setting are as realistic as possible. To see this, suppose that, in the CP auction, the bidders' costs are drawn from the truncated normal distributions but in reality they come from uniform distributions. Let $F_i^{CP}$ denote the truncated normal distribution function with support $[\underline{c}, \bar{c}]$ (according to Equation~\eqref{eq:domain_common_priors_approximation}), and let $F_i^{DMP}$ denote the uniform distribution function with support $[\underline{c}_i, \bar{c}_i]$ for all $i\in N$. Then, as shown in the previous section, $[\underline{c}_i,\bar{c}_i]\subset [\underline{c}, \bar{c}]$. Hence, there exists $c\in [\underline{c}, \bar{c}]$ such that $c\in [\underline{c}_i, \bar{c}_i]$ for some $i\in N$; that is, a bidder is allowed to submit a cost lying outside their actual support.

\begin{table}[t]
  \caption{Expected prices and \emph{ex ante} expected utilities for the considered bidding scenario}
  \vspace{0.5cm}
  \begin{tabular*}{0.5\columnwidth}[L]{@{\extracolsep{\fill}}r c c c}
    \hlx{vhv}
    & \textbf{Expected}   & \multicolumn{2}{c}{\textbf{\emph{ex ante} expected utility}, $\Pi_i$}\\
    & \textbf{price}, $p$ & \textbf{Bidder 1} & \textbf{Bidder 2}\\
    \hlx{vhv}
    \textbf{DMP} & $0.583$ & $0.183$ & $0.030$\\
    \textbf{CP} & $0.573$ & $0.176$ & $0.026$\\
    \hlx{vhs}
  \end{tabular*}
  \label{tab:test_results_approximation}
\end{table}

By way of example, consider the numerical example from the previous section. Table~\ref{tab:test_results_approximation} presents the resulting expected prices and \emph{ex ante} expected utilities for both bidders for both auctions. It is difficult to judge by the values of expected prices and \emph{ex ante} expected utilities how erroneous the approximation for each bidder is. To account for this fact, the relative error in expected prices is defined as
\begin{equation}
  \label{eq:relative_error_price_approximation}
  \eta_p = \left|\frac{p^{DMP} - p^{CP}}{p^{DMP}}\right|
\end{equation}
and the relative error in \emph{ex ante} expected utilities as
\begin{equation}
  \label{eq:relative_error_utility_approximation}
  \eta_{\Pi_i} = \left|\frac{\Pi_i^{DMP} - \Pi_i^{CP}}{\Pi_i^{DMP}}\right|
\end{equation}
for all $i\in N$, where $p^{DMP}$ and $p^{CP}$ denote the expected prices for DMP and CP auction respectively, and $\Pi_i^{DMP}$ and $\Pi_i^{CP}$ denote the \emph{ex ante} expected utilities for bidder $i$ for DMP and CP auction respectively. For the values of expected prices and \emph{ex ante} expected utilities depicted in Table~\ref{tab:test_results_approximation}, the (percentage) relative errors are summarised in Table~\ref{tab:test_relative_errors_approximation}.

\begin{table}[t]
  \caption{Percentage relative errors in expected prices and \emph{ex ante} expected utilities for the considered bidding scenario}
  \vspace{0.5cm}
  \begin{tabular*}{0.5\columnwidth}[L]{@{\extracolsep{\fill}}r c c c}
    \hlx{vhv}
    & \textbf{Expected}   & \multicolumn{2}{c}{\textbf{\emph{ex ante} expected utility}, $\Pi_i$}\\
    & \textbf{price}, $p$ & \textbf{Bidder 1} & \textbf{Bidder 2}\\
    \hlx{vhv}
    \textbf{Percentage relative} & \multirow{2}{*}{$1.72\%$} & \multirow{2}{*}{$3.83\%$} & \multirow{2}{*}{$13.33\%$}\\
    \textbf{error, $\eta\cdot 100\%$} & & & \\
    \hlx{vhs}
  \end{tabular*}
  \label{tab:test_relative_errors_approximation}
\end{table}

In this section, it was shown how a DMP bidding scenario can be cast into CP setting by approximating bidders' cost distributions with truncated normal distributions with common support. Furthermore, expected prices and \emph{ex ante} expected utilities were suggested as metrics for quantifying the accuracy of approximating DMP auction with CP auction. In what follows, the proposed metrics are used to study approximation results in four different bidding scenarios with two, three, four and five bidders respectively.
% subsection methodology (end)

\section{Approximation Results} % (fold)
\label{sec:approximation_results_approximation}
This section analyses the results for four bidding scenarios: with $n=2$, $n=3$, $n=4$ and $n=5$ bidders respectively. The discussion concentrates on only up to five bidders due to the following three reasons. Firstly, the time required to simulate the problem increases exponentially with each additional bidder. It should be noted that the simulations were run on a 12-core Xeon processor, and were fully parallelised (i.e., each repetition was run in a separate process, and up to 20 processes were running at any one time). The time required to complete each simulation run took approximately: 1.6 hours for $n=2$ bidders, 25.6 hours for $n=3$ bidders, 76.2 hours for $n=4$ bidders, and 271.9 hours for $n=5$ bidders. Figure~\ref{fig:simulation_time_approximation} depicts results of fitting an exponential function of the form
\begin{equation}
  f(x) = ae^{bx} + c, \quad\textrm{where}\: a,b,c\in\mathbb{R}, \textrm{ and } x\in\mathbb{R}_+
\end{equation}
to the data. Assuming the growth rate of simulation times with each additional bidder will be at least as big as inferred from the measured simulation times, simulating for more than five bidders is impractical. For example, a predicted simulation time for ten bidders is approximately 124,255 hours, which equates to more than 14 years.

\begin{figure}[t!]
  \includegraphics[width=\figsize]{Approximation/Figures/simulation_time}
  \caption{Exponential function fitted to the simulation time data}
  \label{fig:simulation_time_approximation}
\end{figure}

Secondly, as shown by Fibich and Gavish~\cite{FibichGavish2011}, FSM method becomes numerically unstable for large numbers of bidders (cf.~Corrolary~3.2 in \cite{FibichGavish2011}). It applies to EFSM method since it is based on the FSM method.

Finally, it can be noted that since the UK market is currently dominated by an oligopoly of four incumbent network operators (bidders) who own their infrastructure (EE, Vodafone, O2, and Three), solving the problem for up to five bidders is directly relevant.

The procedure for generating the approximation results is as follows:
\begin{enumerate}
\item \annotate{C5.12}{For each chosen value of price weight, generate 100 reputation ratings vectors, $(r_1,\ldots, r_n)$. Each vector is ordered; that is, $r_1 < r_2 < \cdots < r_n$. Therefore, in what follows, bidder 1 is characterised by the lowest reputation rating, bidder 2 by the second lowest, and so on. By ordering individual reputation ratings within the vectors, the mean relative errors in \emph{ex ante} expected utilities can be explored for individual bidders characterised by the lowest reputation rating, second lowest, etc. In other words, if a bidder is characterised by the lowest reputation rating, the mean relative error in \emph{ex ante} expected utility the bidder is going to incur by bidding according to the equilibrium bidding strategies prescribed by the CP auction is quantified. Without this assumption, the mean relative error curves would converge on the same value for all bidders, and thus, some valuable insight into the extent of the mean relative errors in \emph{ex ante} expected utilities would be lost. It is worth noting, however, that the mean relative error in expected price is unaffected by ordering of the reputation ratings.}

\annotate{}{Furthermore, each $r_i$ for each bidder $i$ is drawn from a uniform distribution over the range $(0,1)$. It should be noted that reputation ratings have to be unique: if $(r_1,r_2)$, then $r_1\neq r_2$; and if $r=(r_1,r_2)$ and $g=(g_1,g_2)$ are two consecutively generated reputation rating vectors, then it is required $r \neq g$. By Assumptions~\ref{ass:assumptions_generic_indirect}, there exists at least one $r_i\neq r_j$ for all $1\leq i,j\leq n$ such that $i\neq j$. This immediately rules out the possibility of bidders having equal reputation ratings in case of 2 bidders. In case of 3 or more bidders, Assumptions~\ref{ass:assumptions_generic_indirect} permit for 2 or more bidders (but not all) to be characterised by equal reputation ratings. In order to keep the analysis numerically tractable, however, the bidding scenarios with bidders characterised by equal reputation ratings in case of 3 or more bidders are not considered.}
\item \annotate{}{For each reputation ratings vector, evaluate relative errors in expected price and \emph{ex ante} expected utility per bidder using Equations~\eqref{eq:relative_error_price_approximation}~and~\eqref{eq:relative_error_utility_approximation}.}
\item Evaluate mean relative errors in expected price and \emph{ex ante} expected utility per bidder, and associated 95\% confidence intervals. The confidence interval for the mean is computed using the formula described in~\cite{LawChapter42007}; that is, given a random sample of size $k$ with unknown mean and standard deviation, the confidence interval is defined as
\begin{equation}
  \text{ci}=\bar{X}\pm t_{1-\sfrac{\alpha}{2},k-1}\frac{s}{\sqrt{k}},
\end{equation}
where $\bar{X}$ is the sample mean, $s$ is the sample standard deviation, and $t_{1-\sfrac{\alpha}{2}, k-1}$ is the upper $1-\sfrac{\alpha}{2}$ critical value for the \emph{t}-distribution with $k-1$ degrees of freedom. It is worth noting that for 95\% confidence interval, $\alpha=0.05$.
\item Repeat for price weight values ranging from $0.55$ to $0.99$. Since only feasible bidders are considered, it is required that $w\in(0.5,1)$ which was shown to be sufficient to warrant feasible bidding in Section~\ref{sec:numerical_analysis_indirect}, Chapter~\ref{cha:indirect}.
\end{enumerate}

\subsection{$n=2$ Bidders} % (fold)
\label{sub:n_2_bidders_approximation}
The approximation results for two bidders are depicted in Figure~\ref{fig:compare_2_bidders_approximation}. It is worth observing that as the price weight increases, the confidence intervals for the mean relative errors decrease. This is a direct consequence of the fact that as the price weight approaches 1, the actual values of the reputation ratings of the bidders do not significantly influence the mean relative errors in expected price and \emph{ex ante} expected utilities for both bidders. To see this, recall from Equation~\eqref{eq:domain_common_priors_approximation} the common support $[\min_i{\underline{c}_i}, \max_i{\bar{c}_i}]=[(1-w)\min_i{r_i}, (1-w)\max_i{r_i} + w]$. As $w\to 1$, this reduces to $[\lim_{w\to 1}(1-w)\min_i{r_i}, \lim_{w\to 1}(1-w)\max_i{r_i} + w] = [0,1]$. Hence, as the price weight increases, the less significant the effect of the reputation ratings on the common support.

Another interesting observation is that, as the price weight approaches 1, the mean relative errors in \emph{ex ante} expected utilities for both bidders start to converge. This is due to the fact that, as $w$ approaches 1 and in particular at $w=1$, the DMP auction becomes a standard FPA auction with all bidders characterised by uniform distributions which are overlapping to a high degree; i.e., with some abuse of notation, $F_i(x)\approx F_j(x)$ for all $x$, $i\neq j$ and $i,j\in N$. The same is true for the CP auction with this difference that all bidders are characterised by almost equal truncated normal distributions. Furthermore, in both auctions, the bidders are characterised by symmetric, albeit different across auctions, equilibrium bidding strategies. This is due to the fact that at a symmetric equilibrium the support becomes identical in both auctions, and hence, uniform distribution of costs and truncated normal distribution of costs will result in different equilibrium bidding strategies. This in turn leads to almost equal mean relative errors in \emph{ex ante} expected utilities for all bidders.

The error bounds are explored next. The mean error in expected prices is approximately linearly increasing in price weight, and is bounded from above by 8\% and from below by 3\%. The mean error in \emph{ex ante} expected utility for bidder 1 also linearly increasing in price weight, and is bounded from above by 15\% and from below by 7\%. For bidder 2, however, the relationship between the price weight and the mean error is nonlinear, with the error attaining its maximum of approximately 15.5\% for the price weight of $w\approx 0.8$. It is bounded from above by 15.5\% and from below by 13\%. It is clear that bidder 1 who is characterised by lower reputation rating is experiencing overall smaller mean error for all values of the price weight. However, as $w\to 1$ and as explained in the previous paragraph, the mean error converges on the same value of approximately 15\% for both bidders.

To summarise, for all analysed values of price weight, the mean relative error in expected prices is relatively small compared to the mean errors in \emph{ex ante} expected utilities for both bidders. It is important to notice that the mean relative errors for both bidders are bounded from above by the same mean relative error of 15\%. In terms of the lower bound, however, bidder 2 who is characterised by higher reputation rating is characterised by much higher mean relative error (13\% in contrast to \emph{only} 7\% for bidder~1).
% subsection n_2_bidders_approximation (end)

\begin{figure}[p!]
  \includegraphics[width=\figsize]{Approximation/Figures/compare_2_bidders}
  \caption{Approximation results for two bidders}
  \label{fig:compare_2_bidders_approximation}
  \vspace{10mm}
  \includegraphics[width=\figsize]{Approximation/Figures/compare_3_bidders}
  \caption{Approximation results for three bidders}
  \label{fig:compare_3_bidders_approximation}
\end{figure}

\subsection{$n=3$ Bidders} % (fold)
\label{sub:n_3_bidders_approximation}
Figure~\ref{fig:compare_3_bidders_approximation} depicts the approximation results for three bidders. First of all, it should be noted that the first two observations pointed out in case of two bidders also apply to the current case. More specifically, as the price weight increases, the confidence intervals for the mean relative errors decrease, and, as the price weight approaches 1, the mean relative errors in \emph{ex ante} expected utilities for all bidders start to converge.

All mean relative errors, unlike in the case of two bidders, exhibit clear nonlinearity in price weight. Furthermore, the mean relative error in expected prices is nondecreasing as the price weight increases, and achieves its maximum at $w = 0.99$. It is bounded from above by 5\% and from below by approximately 1.8\%. The mean relative error in \emph{ex ante} expected utilities for bidder 1 is bounded from above by 10\% and from below by 4.5\%. The mean relative error in \emph{ex ante} expected utilities for bidder 2 is also bounded from above by 10\%, but it is bounded from below by 7\%. It is worth noting that the shape of the mean relative error curve for bidder 2 resembles that of the mean relative error curve for bidder 1 translated in y-direction. Finally, the mean relative error in \emph{ex ante} expected utilities for bidder 3 is bounded from above by 15\% and from below by 10\%.

As expected, bidder 3 who is characterised by the highest reputation rating experiences the highest mean relative error in \emph{ex ante} expected utilities for all values of the price weight out of all bidders. In fact, the lower bound for bidder 3 is the same as the upper bound for the remaining bidders. This agrees with the conclusion drawn for the case of two bidders, where bidder 2 was the bidder characterised by the highest reputation rating and experienced the highest mean relative error out of all bidders.
% subsection n_3_bidders_approximation (end)

\begin{figure}[p!]
  \includegraphics[width=\figsize]{Approximation/Figures/compare_4_bidders}
  \caption{Approximation results for four bidders}
  \label{fig:compare_4_bidders_approximation}
  \vspace{10mm}
  \includegraphics[width=\figsize]{Approximation/Figures/compare_5_bidders}
  \caption{Approximation results for five bidders}
  \label{fig:compare_5_bidders_approximation}
\end{figure}

\subsection{$n=4$ Bidders} % (fold)
\label{sub:n_4_bidders_approximation}
Figure~\ref{fig:compare_4_bidders_approximation} depicts the approximation results for four bidders. Firstly, it should be noted that, similarly to the previous two scenarios, as the price weight approaches 1, the mean relative errors in \emph{ex ante} expected utilities for all bidders start to converge. Furthermore, as the price weight increases, the confidence intervals for the mean relative errors decrease.

In terms of shape, similarly to the case of three bidders, all mean relative errors exhibit nonlinearity in price weight. Furthermore, the mean relative error in expected prices is bounded from above by approximately 0.7\%, and from below by approximately 0.1\%. The mean relative error in \emph{ex ante} expected utilities for bidder 1 is bounded from above by 2\%, and from below by approximately 0.1\%. The mean relative error in \emph{ex ante} expected utilities for bidder 2 is bounded from above by 2.5\%, and from below by 0.1\%. It is worth noting that, for the values of price weight $w\in [0.65, 0.9]$, the mean relative error for bidder 2 is actually smaller than for bidder 1, even though bidder 1 is characterised by the lowest reputation rating. The mean relative error in \emph{ex ante} expected utilities for bidder 3 is bounded from above by 6.1\%, and from below by 1.8\%. Finally, the mean relative error for bidder 4 is bounded from above by 16\%, and from below by 1.5\%.

As expected, bidder 4 who is characterised by the highest reputation rating experiences the highest mean relative error in \emph{ex ante} expected utilities for all values of the price weight out of all bidders. This agrees with the conclusion drawn for the previous two bidding scenarios, where bidder who was characterised by the highest reputation rating, experienced the highest mean relative error out of all bidders. It should further be noted that the range of values the mean relative error takes is much larger than it was the case for bidders characterised by the highest reputation rating in the previous two bidding scenarios.
% subsection n_4_bidders_approximation (end)

\subsection{$n=5$ Bidders} % (fold)
\label{sub:n_5_bidders_approximation}
Figure~\ref{fig:compare_5_bidders_approximation} depicts the approximation results for five bidders. Similarly to the previous three scenarios, as the price weight approaches 1, the mean relative errors in \emph{ex ante} expected utilities for all bidders start to converge. Furthermore, as the price weight increases, the confidence intervals for the mean relative errors decrease.

All mean relative errors, similarly to the case of three and four bidders, exhibit clear nonlinearity in price weight. Furthermore, the mean relative error in expected prices is bounded from above by approximately 6\%, and from below by approximately 1\%. The mean relative error in \emph{ex ante} expected utilities is bounded from above and below by: 10\% and 3\% respectively for bidder 1; 9\% and 3.5\% for bidder 2; 8\% and 3\% for bidder 3; 9.5\% and 3\% for bidder 4; 16.5\% and 2\% for bidder 5.

Similarly to the previous scenarios, bidder 5 who is characterised by the highest reputation rating experiences the highest mean relative error in \emph{ex ante} expected utilities; however, unlike in the previously considered scenarios, it no longer holds for \emph{all} values of the price weight. In particular, while for the values of price weight $w\in [0.55, 0.7]$ bidder 5 is indeed characterised by the highest error, for $w\in [0.8, 1)$ the error is the lowest. Interestingly, it is bidder 1 who is characterised by the highest relative error for $w\in [0.8, 1)$. This is an unexpected result as it contradicts the conclusions drawn from the previously considered scenarios. At the same time, it is a positive result for bidder 5 as it means that being the bidder characterised by the highest reputation rating does not necessarily entails experiencing the highest error for all values of the price weight. Finally, similarly to previous scenarios, bidder 5 is still characterised by largest range of mean relative errors out all bidders.
% subsection n_5_bidders_approximation (end)

\begin{figure}[p!]
  \includegraphics[width=\figsize]{Approximation/Figures/compare_price}
  \caption{Mean relative error in expected prices across all bidding scenarios}
  \label{fig:compare_price_approximation}
  \vspace{10mm}
  \includegraphics[width=\figsize]{Approximation/Figures/compare_bidder_1}
  \caption{Mean relative error in \emph{ex ante} expected utilities for bidder 1 across all bidding scenarios}
  \label{fig:compare_bidder_1_approximation}
\end{figure}
\begin{figure}[p!]
  \includegraphics[width=\figsize]{Approximation/Figures/compare_bidder_2}
  \caption{Mean relative error in \emph{ex ante} expected utilities for bidder 2 across all bidding scenarios}
  \label{fig:compare_bidder_2_approximation}
  \vspace{10mm}
  \includegraphics[width=\figsize]{Approximation/Figures/compare_bidder_3}
  \caption{Mean relative error in \emph{ex ante} expected utilities for bidder 3 across all bidding scenarios}
  \label{fig:compare_bidder_3_approximation}
\end{figure}
\begin{figure}[t!]
  \includegraphics[width=\figsize]{Approximation/Figures/compare_bidder_4}
  \caption{Mean relative error in \emph{ex ante} expected utilities for bidder 4 across all bidding scenarios}
  \label{fig:compare_bidder_4_approximation}
\end{figure}

\subsection{Discussion} % (fold)
\label{sub:discussion_approximation}
Considering all bidding scenarios together, the relationship between the number of bidders and mean relative errors in prices and \emph{ex ante} expected utilities for each bidder is nonlinear. Furthermore, there is no clear tendency between the number of bidders and mean relative errors; that is, the mean relative errors do not necessarily decrease with each additional bidder. To see this, it can be noted that for all values of the price weight, the mean relative error in expected prices decreases for 2 and 3 bidders, achieves its minimum for 4 bidders, and then increases for 5 bidders (see Figure~\ref{fig:compare_price_approximation}). The same applies to the mean relative errors in \emph{ex ante} expected utilities for bidder 1 and 2 (see Figures~\ref{fig:compare_bidder_1_approximation} and \ref{fig:compare_bidder_2_approximation}). However, the situation is more complicated for bidder 3 and bidder 4. In the former case, for the price weight values $w\in[0.55, 0.62]$, the mean relative error decreases with each additional bidder, while for $w\in(0.62, 1)$ it behaves similarly to the case of bidder 1 and 2 (see Figure~\ref{fig:compare_bidder_3_approximation}). In the latter case, the situation is almost exactly the same: the mean relative error decreases with each additional bidder for $w\in[0.55, 0.82]$, and behaves similarly to the case of bidder 1 and 2 for $w\in(0.82, 1)$ (see Figure~\ref{fig:compare_bidder_4_approximation}).

It can be concluded, however, that approximating the network selection mechanism employed by the DMP with a CP auction consitutes a valid alternative, and as such, even though not perfectly accurate (mean relative errors as large as almost 16\% for all bidders), it might be a more desirable option for the network operators due to the wealth of numerical methods available that have been extensively studied by the researchers~\cite{HubbardPaarsch2011}. The same cannot be said about the EFSM method presented in this thesis (Section~\ref{sec:extended_numerical_analysis_indirect}, Chapter~\ref{cha:indirect}), which, first of all, becomes numerically unstable for large number of bidders~\cite{FibichGavish2011}, and secondly, to the best of author's knowledge, has not yet been considered by the economic community.
% subsection discussion (end)
% subsection results (end)
% section network_selection_mechanism_cast_into_common_priors_setting (end)

\section{Summary} % (fold)
\label{sec:summary_approximation}
In this chapter, it was explored whether the DMP auction can be approximated with a CP auction. To this end, the notion of the CP auction was introduced and formally defined. It was shown that the pure strategy Bayesian Nash equilibrium exists and is unique, provided the cost distributions for each bidder satisfy certain assumptions (see Proposition~\ref{prop:characterization_of_the_equilibrium_in_common_priors_setting_approximation}).

Furthermore, this chapter presented a numerical algorithm, first proposed by Bajari~\cite{Bajari2001a}, for numerically approximating equilibrium bidding strategies to the CP auction (see Algorithm~\ref{alg:forward_shooting_method_approximation}). The method was verified for correct implementation in two ways: by comparing the resultant equilibrium bidding strategy functions with those presented by Bajari~\cite{Bajari2001a}; and by testing the equilibrium bidding strategy functions for sufficiency condition for a pure strategy Bayesian Nash equilibrium.

Finally, the DMP auction was modelled as a CP auction where each bidder drew their costs from a truncated normal distribution with common support but differing parameters. A formal methodology for comparing the results generated by the CP auction with those of the DMP auction was presented. The methodology is based on two metrics: expected price and \emph{ex ante} expected utilities for all bidders. The chapter culminated with the analysis of approximation errors in four bidding scenarios: with $n=2$, $n=3$, $n=4$ and $n=5$ bidders. It was concluded that, even though not perfectly accurate (approximation errors as large as 16\% for all bidders), approximating the original DMP auction with the CP auction might be a more desirable option for the network operators. This is emphasised by the fact that there exists an abundance of numerical methods for solving a CP auction that have been extensively studied by the economic community, which are less prone to numerical errors than the FSM method and its derivatives such as the EFSM method.
% section summary_approximation (end)
