\documentclass[10pt,a4paper,notitlepage]{article}
\usepackage{amsmath,amsfonts,amsthm,amssymb,graphicx,url,fancyhdr,fancybox,fancyvrb}
\usepackage[intoc]{nomencl}
\usepackage{rotating}
\usepackage[margin=10pt,font=small,labelfont=bf]{caption}
\usepackage{subcaption}
\usepackage{wrapfig}
\usepackage{multirow}
\usepackage[nottoc]{tocbibind}
\usepackage{fullpage}
\usepackage{rotating}
\usepackage{xfrac}
\numberwithin{equation}{section}

%%%---Set line spread------------------------------------------------------------------
\linespread{1.5}
\pagestyle{plain}

% PDF SETUP------FILL IN HERE THE DOC TITLE AND AUTHOR---------------------------------
\usepackage[
      bookmarks,
      hypertexnames=false,
      breaklinks=true,
      pdftitle={List of Corrections},
      pdfauthor={Jakub Konka},
      plainpages=false]{hyperref}

\usepackage{parskip}

%%%---Start document-------------------------------------------------------------------
\begin{document}

\title{List of Corrections: PhD Viva}
\date{\today}
\author{Jakub Konka}

\maketitle

%% General
\setcounter{section}{0}
\renewcommand{\thesection}{G}
\renewcommand{\thesubsection}{G\arabic{subsection}}
\section{General Comments}

\setcounter{subsection}{0}
\subsection{Comment 1}
The presented work is not in the University Thesis Style---change. Thesis should conform to SU style.

\textbf{Response:}
The style was changed to match the University Thesis Style. More specifically:
\begin{enumerate}
	\item Title page now features University and Department;
	\item Lines are one-half spaced;
	\item Page with declaration of originality and ownership now features a heading;
\end{enumerate}

\subsection{Comment 2}
Section 2.2---you have slipped into first person English–––this should be changed throughout the entire thesis. Convert to third person English throughout. We/our.

\textbf{Response:}
All occurrences of ``we'' or ``our'' were removed and replaced with their passive voice counterpart.

\subsection{Comment 3}
Proofs should be in an appendix, not at the end of the chapter.

\textbf{Response:}
All proofs were moved to Appendix A. For convenience, each proposition proved in the Appendix is restated before the actual proof is conducted (numbering of propositions was preserved). Furthermore, in the thesis, after the first proposition is stated, the reader is directed to the Appendix for the proof.

\subsection{Comment 4}
Every equation should have a number.

\textbf{Response:}
All equations are now numbered.

\clearpage

%% Chapter 1
\setcounter{section}{0}
\renewcommand{\thesection}{C\arabic{section}}
\renewcommand{\thesubsection}{C\arabic{section}.\arabic{subsection}}
\section{Chapter 1 Comments}
\subsection{Comment 1}
Objective 3---why simple? A model that facilitates... would be more appropriate.

\textbf{Response:}
Removed ``simple'' altogether.

\subsection{Comment 2}
What is a bearer service?

\textbf{Response:}
Bearer service, in telecommunications, is equivalent to data service; that is, a service that allows transmission of information between network interfaces. ``Bearer'' was removed from the text. Instead, some examples of services were added.

\subsection{Comment 3}
Are network operators interested in efficient network usage? Revenue generation may be stronger motivation?

\textbf{Response:}
I agree that revenue generation will be a stronger motivating factor for network operators, but at the same time, efficient network usage should not dismissed.

\subsection{Comment 4}
You do not make clear at the end of section 1.1 how the work conducted will be used (either in thesis or wider industry) nor how the contributions to knowledge will be gained.

\textbf{Response:}
A short paragraph outlining this has now been added.

\subsection{Comment 5}
Is your work reported under Contribution 2 being used by anybody else? Chapter 1 needs more info on the DMP at this point in the thesis. Is it used today, is it just a theory (a concept that is not used in practice)?

\textbf{Response:}
More info on the DMP has been added.

\subsection{Comment 6}
How is the work identified under your main contributions different to what is already in the public domain? You do not say and it is not clear from the presentation. Will the economic analysis of the network selection in DMP be scalable beyond this construct?


\textbf{Response:}
This has now been explained.

\subsection{Comment 7}
I am not fully clear as to what other work is going on in this area based on the work presented in this thesis---I appreciate that some of this will come in subsequent chapters but a summary should be included in Chapter 1.

\textbf{Response:}
A short summary has been incorporated into the Main Contributions section.

\clearpage

%% Chapter 2
\section{Chapter 2 Comments}
\subsection{Comment 1}
Weird font sizes used in Fig.~2.3.

\textbf{Response:}
Unified font sizes.

\subsection{Comment 2}
Is the flow of network traffic important for the diagram in Fig.~2.1? Fig.~2.1: Ensure diagram is bidirectional.

\textbf{Response:}
Replaced ``lighting bolts'' with double pointed arrows in Fig.~2.1.

\subsection{Comment 3}
The first two paragraphs of section 2.1.3 are not needed.

\textbf{Response:}
Removed.

\subsection{Comment 4}
Section 3.2.1---auction theory is promising for what?

\textbf{Response:}
Auction theory is one of the most \emph{prominent} branches of economics.

\subsection{Comment 5}
Need a 4th level of headings.

\textbf{Response:}
Modified accordingly.

\subsection{Comment 6}
Will the Internet always be the source as identified in Fig. 2.1?

\textbf{Response:}
Yes. To make it clear, modified the reference of Fig.~2.1 so that it now references a similar figure in [4].

\subsection{Comment 7}
Why is the telecoms industry pushing to have an all IP based system?

\textbf{Response:}
Rephrased to explain the drive for an all-IP-based platform in more detail.

\subsection{Comment 8}
You need to add a definition of utility in section 2.2.1 (or similar) and also that it can be difficult to quantify the utility.

\textbf{Response:}
The section on utility-based network selection was rewritten. It now features definition of utility function and an explanation of why it is difficult to quantify it.

\subsection{Comment 9}
There is an over reliance on reference [11] in this chapter.

\textbf{Response:}
The literature review part of the chapter was substantially rewritten.

\subsection{Comment 10}
So at the end of section 2.2.2 what point are you trying to make? Is this useful, has anybody used this in practice? What are the strengths and weaknesses?

\textbf{Response:}
Concluding remarks about the difficulties of the MADM-based schemes were added to Section 2.2.2.

\subsection{Comment 11}
Why do you concentrate on games between subscribers and games between network operators, and not between operators and subscribers? P13: Why not a mixed game? Need a better justification. Are there advantages of approaching the problem as a mixed game?

\textbf{Response:}
A detailed explanation of why mixed games are not desirable was added.

\subsection{Comment 12}
Why is the information you provide on games between subscribers and games between operators useful to your thesis? At the end of each of these unnumbered subsections (latex problem?) you do not say. Is it good or bad? What are the gaps in the research? How does it impact on your thesis? Some sections are merely descriptive---they need a point, e.g., Games Between Operators.

\textbf{Response:}
Remedied by commenting on the approaches of other researchers and their deficiencies.

\subsection{Comment 13}
Subsection 2.2.4---I agree that there are many different approaches on network selection but you do not give a very comprehensive literature review to support this. Furthermore, you do not make clear why you are dismissing [38 to 42] + other.

\textbf{Response:}
The literature review in Subsection 2.2.4 has now been substantially extended.

\subsection{Comment 14}
You do not make clear on P15 how your work is different to that reported in [8, 28], just that it is---this is missing. Chapter 2 needs to make clear how this work extends the economic aspects of previously published work of other researchers.

\textbf{Response:}
The deficiencies of the research of other researchers were outlined, and it was made clearer how this work extends the work of previously published work.

\clearpage

%% Chapter 3
\section{Chapter 3 Comments}
\subsection{Comment 1}
``report'' is a poor choice of variable.

\textbf{Response:}
Replaced with $\rho$.

\subsection{Comment 2}
P24: mention the paper via specific reference.

\textbf{Response:}
Rephrased and referenced the paper in question.

\subsection{Comment 3}
Section 3.2.2 contains too much design aspects---only reporting with relevance to the thesis is required. You can develop design choices elsewhere.

\textbf{Response:}
The design aspects of the reputation rating systems were added into the thesis after the reviewers of the IEEE Transactions on Vehicular Technology paper pointed out that this was missing. Furthermore, it is more descriptive than anything else, as the reputation rating system was developed by other researchers studying the DMP.

\subsection{Comment 4}
P8: More discussion on the definition of reputation is required. Does the reputation relate to the network operator or to each service offered by the network operator?

\textbf{Response:}
Added explanation in the paragraph describing the reputation rating system.

\subsection{Comment 5}
Is DMP a recognised term? Is the DMP used operationally or just a theoretical concept? Who developed the DMP concept?

\textbf{Response:}
No, it is a theoretical framework for management of communications services in a heterogeneous wireless environment. The concept was first proposed by Irvine~\emph{et al.}~in 2000. The answer to the questions were appropriately incorporated into the thesis.

\subsection{Comment 6}
Why do concentrate on the simplest business model? Why not the most appropriate?

\textbf{Response:}
Changed ``simplest'' to ``basic''. Furthermore, added an explanation as to why the basic business model is the most appropriate.

\subsection{Comment 7}
More discussion on practicalities of substituting subscriber with service provider.

\textbf{Response:}
A discussion was added.

\subsection{Comment 8}
It is not clear how to read Fig.~3.1 with respect to contracts.

\textbf{Response:}
In order to clarify the diagrams, ``Contractual relationship'' was renamed to ``Business relationship'' in Figures~3.1 and 3.2.

\subsection{Comment 9}
Contracts are missing from Fig.~3.2.

\textbf{Response:}
Made the figure more explicit by including subscriber and market provider, and all of the relationships between different entities.

\subsection{Comment 10}
References to common types of auction.

\textbf{Response:}
Added references to each paragraph describing common auction types.

\subsection{Comment 11}
End of Section 3.2.2---referencing in big blocks is totally inapproriate---you need to deal with these individually. This approach does not convey understanding. P24: decomposed large groups of references.

\textbf{Response:}
Removed the first block of references altogether since they are elaborated upon in the subsequent section. Decomposed the second block (regarding QoE) into a short paragraph summarising each paper in relation to QoE.

\subsection{Comment 12}
There are no references in subsection 3.2.1 and an overreliance on [8] elsewhere.

\textbf{Response:}
Section 3.2.1 features now more references (a reference for each auction type; see C3.10). Changed some of the [8] to [44] (LeBodic's thesis). These are the main sources of information about fundamental assumptions of the Digital Marketplace.

\subsection{Comment 13}
You do not provide any text on markets where a physical commodity is traded (e.g., electricity, capacity markets)---this is missing from the thesis as a standalone section to bring out the aspect of physical trade that are relevant or otherwise to your work. Need to mention some other markets, e.g., oil, gas, electricity, etc.

\textbf{Response:}
Section 3.3 has been added, and it considers DMP as a commodity market and compares it with the existing electrity markets. In particular, the UK wholesale electricity market is described and similarities between the two are drawn. The ultimate aim of the section is to demonstrate that, using the example of electricity, DMP is definitely able to operate as a market for trading wireless services. 

\subsection{Comment 14}
You need to be clearer on why you have just provided the bidding strategies rather than developing them---so as far as I know from what you have presented in the thesis, the bidding strategies in [8] and [36] could just be random numbers!

\textbf{Response:}
The thesis now features an explanation.

\subsection{Comment 15}
Again taking your work from an economic point of view is one possible approach---why is it the most important?

\textbf{Response:}
A detailed explanation of why economic analysis of the DMP network selection mechanism is now included in the thesis.

\clearpage

%% Chapter 4
\section{Chapter 4 Comments}
\subsection{Comment 1}
Figs. 4.1 and 4.2---add colour plot.

\textbf{Response:}
Revised accordingly.

\subsection{Comment 2}
In Figs. 4.1 and 4.2 is $i$ and $j$ a proxy for 1 and 2? $n=2$ in this section.

\textbf{Response:}
Yes it is. It was revised accordingly; $i$ was renamed to 1 and $j$ to 2 both in the text and in the figures. The changes make it compatible with Section 5.3, Chapter 5.

\subsection{Comment 3}
Equation (4.7): $n$ was previously the number of network operators---is this now varying with $i$?

\textbf{Response:}
To clarify the exposition, the variables were renamed as follows: $m_i$ is now $\zeta_i$, and $n_i$ is now $\eta_i$. See Equation (4.26) in the revised manuscript.

\subsection{Comment 4}
P40: Need to explain $b'$.

\textbf{Response:}
The apostrophe was dropped to simplify the notation.

\subsection{Comment 5}
Eq.~(4.15): Assuming $b(1) = 1$, need to explain the implications of this. P33: does making the assumption $b(1)=1$ have a physical meaning?

\textbf{Response:}
Added explanation in the text.

\subsection{Comment 6}
First equations on P29---will all the operations use the same weights?

\textbf{Response:}
Clarified in the paragraph before the equations.

\subsection{Comment 7}
What happens if a network operator bids below their cost? Need a discussion on practicalities of real bids (incl. loss leader pricing strategy). Does allowing negative bids have a physical interpretation?

\textbf{Response:}
The point was addressed in the thesis. A paragraph explaining the problem and assumptions of game theory was added to the end of section 4.1.

\subsection{Comment 8}
You need to make clear that $b$ contains a symmetric bid/offer pair.

\textbf{Response:}
The notation is now properly explained.

\subsection{Comment 9}
Need a definition of a bid/offer. Relate the maths back to the application.

\textbf{Response:}
The definition of the bid/offer was already provided in Section~3.2.2, Chapter~3. However, to relate the mathematics back to the application, the definition is restated after the bid/offer is formally introduced in Section~4.1, Chapter~4.

\subsection{Comment 10}
You should provide a better description of FPA before embarking on the mathematics. P28 need a quick primer of first-price sealed-bid auction and Bayesian games. Need to mention why it is a Bayesian game of this type.

\textbf{Response:}
The beginning of Section~4.1 now features a more detailed description of FPA and how it relates to Bayesian games.

\subsection{Comment 11}
Summary in Section 4.4 undersells what has been done. Need to relate to the wider, more general problem in the application domain.

\textbf{Response:}
The summary was rewritten and now includes more discussion about the applicability of the results to real life.

\clearpage

%% Chapter 5
\section{Chapter 5 Comments}
\subsection{Comment 1}
Remove obsolete figures from Chapter 5.

\textbf{Response:}
The following figures were removed: 5.8, 5.10, 5.12, 5.14, 5.17 and 5.19. Appropriate changes in the sections referencing the figures were made.

\subsection{Comment 2}
Remove Section 5.4.4 (Chebyshev) entirely.

\textbf{Response:}
Section was removed and appropriate changes in the sections referencing the section were made.

\subsection{Comment 3}
Fig. 5.16---what is customised finite difference? Change to EFSM.

\textbf{Response:}
The caption of the figure was changed to EFSM (Fig. 5.12 in the revised manuscript). Furthermore, any mention of ``customised finite-differences'' method was removed from the text.

\subsection{Comment 4}
P66: ``in particular'' should be ``for example''.

\textbf{Response:}
Revised accordingly.

\subsection{Comment 5}
P65: mode detail on what ``slightly modified'' entails---just say ``using existing methods''.

\textbf{Response:}
Revised accordingly.

\subsection{Comment 6}
P56: should read network operator 1.

\textbf{Response:}
Revised accordingly.

\subsection{Comment 7}
P61: why do you make this statement regarding ``abuse of notation''?

\textbf{Response:}
Originally, it was introduced to justify the weird notation of the expected prices. However, since the notation is legible, the statement ``abuse of notation'' was removed.

\subsection{Comment 8}
P61: How do you know 10,000 is enough? Was this number empirically chosen? Need to justify. Why has 10,000 been chosen? Is it large?

\textbf{Response:}
A sentence explaining why 10,000 observations was added in the paragraph.

\subsection{Comment 9}
Superscript notation in Fig.~5.4 not clear. Put superscripts in brackets so that it does not look like squaring.

\textbf{Response:}
Superscripts are now in brackets in the text and in Fig.~5.4.

\subsection{Comment 10}
Implications of adapting price weights should be discussed.

\textbf{Response:}
A few sentences discussing the implications were added to the appropriate paragraph.

\subsection{Comment 11}
P68, P71, P82---how do these algorithms terminate?

\textbf{Response:}
Each algorithmic listing features ``input'' and ``output'' statements at the very top of the listing.

\subsection{Comment 12}
You have avoided using the $k$-tuple language elsewhere in the thesis, so why start on P69?

\textbf{Response:}
All occurrences of $k$-tuples were changed to vectors.

\subsection{Comment 13}
Weird ends in Figs.~5.12 and 5.13---any explanation?

\textbf{Response:}
The explanation is provided in the text.

\subsection{Comment 14}
Brackets on equation after (5.33) is strange. P81: need clearer notation.

\textbf{Response:}
The set-builder notation was removed altogether. Instead, the conditions were unwrapped into separate equations, and are reference in the algorithms and in the text.

\subsection{Comment 15}
What value are Kaplan and Zamir putting on the non-standard equilibria?

\textbf{Response:}
Added an explanation of the importance of the non-standard equilibria.

\subsection{Comment 16}
P84: is this brevity?

\textbf{Response:}
Rephrased.

\subsection{Comment 17}
What is the algorithm on P58 based on? Is it yours or taken from elsewhere? P58: Make clear whose procedure this is. Put into algorithmic style.

\textbf{Response:}
Added appropriate listing (Listing 5.1). Furthermore, updated the description of the steps.

\subsection{Comment 18}
Can you explain the simplification made before equation (5.27) and its impact?

\textbf{Response:}
A sentence explaining the impact of the simplifaction was added.

\subsection{Comment 19}
Section 5.4.1 you need to make clear what FSM is all about then move to the specifics of the problem of interest. Section 5.4.2---again you need to make clear what PPM is all about then move to the specifics of the problem of interest.

\textbf{Response:}
Removed the first paragraph from the sections which was confusing and delayed the description of the methods under consideration. Moved the paragraph to the end of the section preceding FSM and PPM sections.

\subsection{Comment 20}
Section 5.4---you do not make clear whether these types of equations (5.13) to (5.14) arise elsewhere in other domains. P63: do the ODEs appear anywhere else with the same structure. Perhaps a quick look would be useful.

\textbf{Response:}
To the best of my knowledge, the system of ODEs is unique to auction theory domain. An explanation was added to the thesis to make it clear.

\subsection{Comment 21}
Why have you chosen to mention finite differences on P64? This is just one possibility + you give no reference. 

\textbf{Response:}
Added an explanation and relevant references that justify finite-difference methods.

\subsection{Comment 22}
Why do the FSM and PPM methods suit your problem? Why select FSM? Need to explain how problem naturally fits into FSM. Justify.

\textbf{Response:}
It was made clear in the thesis why the problem naturally fits into the framework of the FSM method. Furthermore, the particular choice of the FSM and PPM is argumented as well.

\subsection{Comment 23}
Clarify that EFSM is my own work. Need to differentiate this work from Lebrun.

\textbf{Response:}
A sentenced highlighting the fact that EFSM method was developed by the author was added to the introductory paragraph of the EFSM method.

\subsection{Comment 24}
A general point for discussion at the viva concerns with the value of the closed form solutions obtained and their applicability in an operational setting. To what extent do the assumptions made to yield these solutions compromise their operational value. This aspect is further compounded as you move to more numerical solutions.

\textbf{Response:}
In Chapter 5, an additional simplifying assumption is made about the DMP auction: costs for the network operators are uniformly distributed. The impact of this impact on the applicability of the solutions to real life is now discussed in the summary section of Chapter 5.

\clearpage

%% Chapter 6
\section{Chapter 6 Comments}
\subsection{Comment 1}
P99: Eq.~6.5 notation issue.

\textbf{Response:}
Made appropriate changes in the captions of figures 6.5 and 6.6.

\subsection{Comment 2}
P100: Last equation. Why is denominator 4? 95\% rule. Mention this first.

\textbf{Response:}
The paragraph explaining the choice of the parameters was rephrased.

\subsection{Comment 3}
What is the point of this chapter? What is the motivation? Need to explain how this chapter will be a contribution over and about the previous chapter.

\textbf{Response:}
The introduction to Chapter 6 was rewritten and features the problem of instability and its possible resolutions explained.

\subsection{Comment 4}
Have you had any problems with convergence of the numerical methods? Need to mention that FSM/EFSM has instability.

\textbf{Response:}
The introduction to Chapter 6 now makes it clear that FSM and EFSM may become unstable for large numbers of bidders.

\subsection{Comment 5}
Why choose a CP that is a truncated normal distribution?

\textbf{Response:}
The motivation for truncated normal distribution is now explained in the thesis.

\subsection{Comment 6}
P91: What is the physical interpretation of common priors in this application? Does the common priors assumption have any physical interpretation in terms of network selection?

\textbf{Response:}
Common priors assumption is used to address the instability issues of the EFSM algorithm. That is, it is used to approximate the DMP auction so that the EFSM algorithm does not have to be employ in deriving approximate equilibrium bidding strategies. However, this in itself might lead to a game between the network operators as now elaborated upon in the thesis.

\clearpage

%% Chapter 7
\section{Chapter 7 Comments}
\subsection{Comment 1}
How credible is it to expect network operators to adopt the work of your thesis? Could these techniques be implemented in a practical setting?

\textbf{Response:}
FIX:ME

\subsection{Comment 2}
The market is never in equilibrium---what does this mean for your work?

\textbf{Response:}
FIX:ME

\subsection{Comment 3}
A range of assumptions have been necessary to gain closed-form solutions (and numerical solutions) but this also narrowed the breadth of application---to what extent does this impact on the usefulness of the generalised result you derive?

\textbf{Response:}
Discussion of the limitations introduced through a range of assumptions was added.

\subsection{Comment 4}
The suggestions for further work are good but are too brief and take no real account of the practical aspect of the work. Further work should explain how results could be broadened and used in practice because this is where the value of the work lies.

\textbf{Response:}
FIX:ME

\end{document} 
