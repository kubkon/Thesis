\chapter*{Abstract} % (fold)
\label{cha:abstract}
\addcontentsline{toc}{chapter}{Abstract}

Mobile communications has become an indispensable part of our everyday lives, with increasingly more people owning a smartphone, and being given access to a plethora of wireless access technologies: WiFi, 3G, and 4G. In an environment of such diversity, where each wireless access technology has its own distinct characteristics, network selection mechanisms provide an efficient way of handling communications services by matching the services' required quality with the characteristics of a particular access technology.

This thesis explores the economic aspects of intelligent network selection. The problem is studied within the context of Digital Marketplace---a market-based framework where network operators compete in a procurement auction-based setting for the right to transport the user's requested service over their infrastructure. The network selection mechanism advocated by the Digital Marketplace lacks extensive and rigorous economic analysis. This thesis addresses this deficiency by providing an extensive game theoretic analysis of the network selection mechanism. The equilibrium bidding strategies are characterised for an arbitrary number of network operators, and explicitly derived for two network operators. Furthermore, three numerical methods are proposed that allow for numerical derivation of the equilibrium bidding strategies in the case of more than two network operators: forward shooting method (FSM), polynomial projection method (PPM), and extended FSM (EFSM). The FSM and PPM methods allow for numerically approximating equilibrium bidding strategies for a subset of all possible bidding scenarios, while the EFSM method enables computation of the numerical solution to all bidding scenarios. Finally, since the EFSM method becomes numerically unstable for large number of network operators, a methodology for approximating the network selection mechanism with an auction format for which there exist many well-defined and extensively studied numerical solutions is discussed. With the results presented in this thesis, should the Digital Marketplace be chosen as a management platform of future wireless access networks, the participating network operators can use the results derived in this thesis to formulate their pricing strategies in the most optimal way, while the users are in the position to understand the prices they will be required to pay.
% chapter abstract (end)
